\documentclass{article}

\usepackage{amsmath,amssymb}
\usepackage{hyperref}

\begin{document}

\title{Number of profiles for different types of schedulers}
\author{Erik Brinkman\\\texttt{\href{mailto:erik.brinkman@umich.edu}{erik.brinkman@umich.edu}}}
\maketitle

Independent of scheduler type, there are a few important quantities for describing the number of profiles of a specific scheduler.
$\mathcal R$, the set of roles in the game.
$n_r > 0$, the reduced number of players in role $r$.
$s_r > 0$, the number of strategies in fully explored set for role $r$.
$d_r \ge 0$, the number of deviating strategies for role $r$.

For normal games, and games with a hierarchical reduction, the closed form is relatively simple.
\begin{equation*}
    \prod_{r \in \mathcal R} {n_r + s_r - 1 \choose n_r} + \sum_{r \in \mathcal R} d_r \prod_{\rho \in \mathcal R} {n_\rho + s_\rho - 1 - \mathbb{I}(r = \rho) \choose n_\rho - \mathbb{I}(r = \rho)}
\end{equation*}
where $\mathbb{I}$ is the indicator function, one when true, zero when false. The first term is the number of profiles in the ``full'' game, and the second term is the number of profiles for the deviations.

For deviation preserving reduction games, the form is a little more complicated.
\begin{equation*}
    \begin{split}
        & \sum_{r \in \mathcal R} s_r \prod_{\rho \in \mathcal R} {n_\rho + s_\rho - 1 - \mathbb{I}(r = \rho) \choose n_\rho - \mathbb{I}(r = \rho)} \\
        & - \sum_{\mathcal P \in \mathbb P(\mathcal R) \setminus \varnothing} \left[ |\mathcal P| - 1 \right] \prod_{r \in \mathcal P} s_r \prod_{r \in \mathcal R \setminus \mathcal P} \left[ {n_r + s_r - 1 \choose n_r} - s_r \right] \\
        & + \sum_{r \in \mathcal R} d_r \prod_{\rho \in \mathcal R} {n_\rho + s_\rho - 1 - \mathbb{I}(r = \rho) \choose n_\rho - \mathbb{I}(r = \rho)} \\
        & + \sum_{r \in \mathcal R} d_r \sum_{\rho \in \mathcal R} s_\rho \prod_{o \in \mathcal R} {n_o + s_o - 1 - \mathbb{I}(r = o) - \mathbb{I}(\rho = o) \choose n_o - \mathbb{I}(r = o) - \mathbb{I}(\rho = o)}
    \end{split}
\end{equation*}
The first term is the number of payoffs in the full game, which is a slight overestimate of the DPR profiles.
The second term is the amount of overestimate correcting the number of full game profiles.
The third term is the number of profiles for the deviators' payoffs.
The last term is the number of profiles for the non-deviators' payoffs when there is a deviator.
Note, that for $|\mathcal R|$ items in the second term have $|P| = 1$, and so are multiplied by 0 and can be omitted when actually computing the summation.

\end{document}
